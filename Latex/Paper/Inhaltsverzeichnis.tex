\documentclass[a4paper,12pt]{article}

% Deutsche Sprache und Umlaute
\usepackage[ngerman]{babel}
\usepackage[utf8]{inputenc}
\usepackage[T1]{fontenc}

% Schönere Schrift und Layout
\usepackage{lmodern}
\usepackage{setspace}
\onehalfspacing

\usepackage[hidelinks]{hyperref}

\begin{document}

Inhaltsverzeichnis:
\begin{itemize}
    \item Abstract (Alle)
    \item Einleitung (Alle)
    \item Theoretischer Hintergrund (Gabriel)
    \item (sub) Definition und Funktionsweise eines Chatbots
    \item (sub) Sozialitische Gestalung eines Chatbots
    \item (sub) Einfluss Visuellegestaltung
    \item Gestaltung des Chatbot Avatar (Jan)
    \item (sub) Typen von Avataren
    \item (sub) Der realitätsgrat der Avataren
    \item (sub) Körpersprache und Gestik
    \item (subsub) Emotion??
    \item (sub) Personalisierung von Avataren
    \item Stimmen Chatbots (Flo)
    \item  Visuelles Interface und Layout gestaltung eines Chatbots (Flo)
    \item (sub) Positionierung des Avatars im Interface
    \item (sub) Layoutstruktur: Chatfenster, Sprechblasen, Benutzerführung
    \item (sub) Farbgestalutung
    \item Preferenzen von Avataren in unterschiedlichen altern (Gabriel)
    \item  Erweiterte und innovative Konzepte //optional
    \item (sub) klassischer 2D-Interfaces
    \item (sub) Augmented Reality (AR) und räumliche Einbettung
    \item Evaluierung (Alle)
    \item Fazit (Alle)
\end{itemize}
\end{document}
