\documentclass[final]{beamer}
\usepackage[T1]{fontenc}
\usepackage[utf8]{inputenc}
\usepackage[ngerman]{babel}

% size=a3, orientation=landscape
% scale=1.4 sorgt dafür, dass die Schrift im Verhältnis zum A3-Blatt groß genug ist
\usepackage[orientation=landscape,size=a3,scale=1.4]{beamerposter}

\usepackage{amsmath}
\usepackage{graphicx}
\usepackage{media9}
\usepackage{tikz}
\usetikzlibrary{arrows.meta}

% Hyperlinks
\hypersetup{colorlinks=true,linkcolor=headerblue,citecolor=headerblue,urlcolor=headerblue}

% Bibliographie - kompakt für Poster
\usepackage[backend=biber,style=numeric-comp,sorting=none,maxnames=1,giveninits=true]{biblatex}
\addbibresource{MeineBibliothek.bib}
\setbeamertemplate{navigation symbols}{}
\setbeamertemplate{headline}{}
% Referenzen kompakter machen, aber DOI als klickbaren Link behalten
\renewcommand*{\bibfont}{\tiny}
\setlength{\bibitemsep}{0pt}
\setlength{\bibnamesep}{0pt}
\setlength{\bibinitsep}{0pt}
\DeclareFieldFormat{title}{}
\DeclareFieldFormat{journaltitle}{}
\DeclareFieldFormat{booktitle}{}
\DeclareFieldFormat{doi}{\href{https://doi.org/#1}{[Link]}}
\DeclareFieldFormat{url}{\href{#1}{[Link]}}
\DeclareFieldFormat{eprint:arxiv}{\href{https://arxiv.org/abs/#1}{[Link]}}
\AtEveryBibitem{
  \iffieldundef{doi}{}{\clearfield{url}} % URL nur zeigen, wenn DOI fehlt
  \clearfield{note}\clearfield{pages}\clearfield{volume}\clearfield{number}
}

% --- Horizontal Bibliography ---
\newcounter{bibcount}
\defbibenvironment{bibliography}
  {\setcounter{bibcount}{0}%
   \raggedright}
  {}
  {\stepcounter{bibcount}%
   \ifnum\value{bibcount}>1 \unspace, \fi % Komma zwischen Einträgen
   \printtext[labelnumberwidth]{%
     \printfield{labelprefix}%
     \printfield{labelnumber}}%
   \addspace}
\renewcommand*{\finentrypunct}{} % Kein Punkt am Ende des Eintrags

% --- Farben des Original-Posters ---
\definecolor{headerblue}{RGB}{0, 102, 178} 
\definecolor{linegrey}{RGB}{220, 220, 220} 

% --- Beamer Style Einstellungen für Lesbarkeit ---
\setbeamercolor{block title}{fg=headerblue,bg=white}
\setbeamercolor{block body}{fg=black,bg=white}
\setbeamerfont{block title}{size=\large,series=\bfseries}
\setbeamerfont{block body}{size=\normalsize}

% Trennlinie (kann noch innerhalb von Spalten genutzt werden, falls nötig)
\newcommand{\posterrule}{\vspace{0.4cm} \hrule height 1pt \color{linegrey} \vspace{0.4cm}}

% --- FOOTER EINSTELLUNG (OPTIMIERT) ---
\setbeamertemplate{footline}{
    % ht= Höhe der Box, dp= Abstand zum unteren Papierrand
    \begin{beamercolorbox}[wd=\paperwidth,ht=4ex,dp=3ex,leftskip=2cm,rightskip=2cm]{structure}
        \color{black}
         % Schrift deutlich vergrößert (statt scriptsize)
        
        % Inhalt mit hfill für perfekte Links-Mitte-Rechts Ausrichtung
        Hochschule Aalen 
        \hfill 
        \today
        \hfill 
        Fortgeschrittene Mensch-Computer-Interaktion
    \end{beamercolorbox}
}

\begin{document}
\begin{frame}[t]

% --- HEADER ---
\begin{columns}[T]
    \begin{column}{.2\textwidth}
        \includegraphics[width=\textwidth]{Hochschule-aalen.svg.png} 
    \end{column}
    \begin{column}{.6\textwidth}
        \centering
        \huge \textbf{Gestaltung von Chatbot-Interfaces und Avataren} \\
        \small
        Jan Herbst (82784) | Gabriel Roth (82798) | Florian Merlau (81775)



    \end{column}
    \begin{column}{.2\textwidth}
        \flushright
        \quad % Platz für QR-Code
    \end{column}
\end{columns}

\vspace{1cm}

% --- HAUPTINHALT ---
\begin{columns}[T]
    % --- SPALTE 1 ---
    \begin{column}{.31\textwidth}
        % Blau markierte Box (Forschungsfrage)
        \setbeamercolor{block title}{fg=white,bg=headerblue}
        \setbeamercolor{block body}{fg=black,bg=gray!10}
        \begin{block}{1 Forschungsfrage}
            Wie sollten Chatbot-Interfaces und Avatare gestaltet werden, um Nutzererfahrung, Vertrauen und Interaktionsqualität zu optimieren?
        \end{block}
        
        \vfill % Dehnt den Raum zwischen den Blöcken
        
        \setbeamercolor{block title}{fg=headerblue,bg=white}
        \setbeamercolor{block body}{fg=black,bg=white}
        
        \begin{block}{2 Definition \& Funktionsweise von Chatbots}
            \begin{itemize}
                \item \textbf{Definition:} Computerprogramme zur dialogischen Kommunikation (textuell/auditiv).
                \item \textbf{Typen:}
                \begin{itemize}
                    \item \textit{Ohne KI:} Starre Menü-Navigation (Regelbasiert).
                    \item \textit{Konversationelle KI:} Verarbeitet Fragen mittels NLU.
                    \item \textit{Generative KI:} Basierend auf LLMs für flexible Inhalte.
                    \item \textit{Virtuelle Agenten:} Kombination aus KI, Deep Learning \& RPA.
                \end{itemize}
                \item \textbf{Entwicklung:} Von einfachen FAQ-Systemen $\rightarrow$ Regelbasierte Systeme $\rightarrow$ KI-Chatbots mit natürlichem Sprachverständnis.
                \item \textbf{Moderne Fähigkeiten:} Kontextverständnis, Fehlerkorrektur (Tippfehler), hohe Vielseitigkeit.
            \end{itemize}
        \end{block}


         \vfill

        

        
        \begin{block}{3 Soziotechnische Gestaltung \& Soziale Signale}
            \begin{itemize}
                \item \textbf{Gestaltung:} Erfolg hängt von technischer Funktionalität UND sozialen Faktoren ab.
                \item \textbf{Soziale Signale:} Werden unterbewusst wahrgenommen und fördern Beziehung.
                \begin{itemize}
                    \item \textit{Typen:} Verbale (Inhalt), Auditive (Laute), Visuelle (Aussehen/Bewegung), Unsichtbare.
                \end{itemize}
                \item \textbf{Visueller Einfluss:} Lächelnde Avatare verlängern Interaktion; Smalltalk prägt Persönlichkeit.
            \end{itemize}
        \end{block}

        \vfill

        \begin{block}{4 Soziale Reaktionen}
            \begin{itemize}
                \item \textit{Social Actor:} Chatbots werden unbewusst als soziale Akteure wahrgenommen.
                \item \textit{Folge:} Nutzer übertragen zwischenmenschliche Verhaltensmuster (z.B. Höflichkeit) auf den Bot.
            \end{itemize}
            
            \vspace{0.2cm}
            \centering
            \includegraphics[width=\linewidth]{soziale_Reaktionen.pdf}
            \par\vspace{0.1cm}
            \textit{\footnotesize Abbildung 1: Soziale Reaktionen. Darstellung nach \cite{Gnewuch2020}.}
        \end{block}

    \end{column}

    % --- SPALTE 2 ---
    \begin{column}{.31\textwidth}
    	


        
        \vfill

        \begin{block}{5 Stimmengestaltung}
            \begin{itemize}
                  \item Je menschlicher eine Computerstimme klingt, desto sympathischer und angenehmer wird sie wahrgenommen.
                  \item \textbf{Erwartung:} Bei einer täuschend echten Stimme erwarten Nutzer oft auch ein intelligenteres Gesprächsniveau.
                  \item \textbf{Lebendigkeit:} Natürliche Schwankungen in der Tonhöhe machen die Stimme freundlicher als eine monotone Sprechweise.
            \end{itemize}
        \end{block}



        \vfill

        \begin{block}{6 Chatbot-Avataren}
            \begin{itemize}
                \item \textbf{Gestaltung:} Beeinflusst Vertrauen, Kompetenz und Wohlbefinden der Nutzer.
                \item \textbf{Kontextabhängigkeit:}
                \begin{itemize}
                    \item \textit{Professionell:} Reduzierte, technische Avatare bevorzugt.
                    \item \textit{Informell:} Menschliche Darstellungen verbessern Eindruck.
                \end{itemize}
            \end{itemize}

            \vspace{0.3cm}
            \centering
            % 2x2 Bilder-Grid verschieben über 7 und Kapitel 7 entfernen
            \begin{tabular}{cc}
                \includegraphics[height=5.5cm]{ChatGPT_logo.svg.png} &
                \includegraphics[height=5.5cm]{Poster/dd.png} \\[0.2cm]
                \includegraphics[height=5.5cm]{Poster/facereal.png} &
                \includegraphics[height=5.5cm]{Poster/Real.png}
            \end{tabular}
            \par\vspace{0.1cm}
            \textit{\footnotesize Abbildung 2: Erscheinungsformen. Eigene Darstellung. \cite{ChatGPTLogoWikipedia}}
            \vspace{0.3cm}
            \begin{itemize}
                \item \textbf{Anthropomorphismus:} Übertragung menschlicher Eigenschaften auf Avatare erzeugt Nähe und Vertrauen.
                \item \textbf{Realitätsgrad:} Beeinflusst Nutzererfahrung maßgeblich.
                \item \textbf{Uncanny Valley:} Fast-menschliche Avatare wirken oft unheimlich oder irritierend.
                \item \textbf{Erwartungshaltung:} Hochrealistische Avatare wecken Erwartungen an echte menschliche Interaktion.
            \end{itemize}
        \end{block}
  
    \end{column}

    % --- SPALTE 3 ---
    \begin{column}{.31\textwidth}
        \begin{block}{7 Interface-Positionierung}
        \begin{itemize}
            \item \textbf{Fokus:} Text/Steuerung primär, Avatar sekundär.
            \item \textbf{Empfehlung:} Avatar peripher (seitlich) statt zentral/dominant.
            \item \textbf{Risiko:} Zentral + Blickkontakt $\rightarrow$ Ablenkung/sozialer Druck.
        \end{itemize}
        \vspace{0.15cm}
        \centering

            
            \vspace{0.2cm}
            \centering
            \includegraphics[width=0.7\linewidth]{Poster/Avatare.pdf}
            \par\vspace{0.1cm}
            \textit{\footnotesize Abbildung 3: Interface-Positionierung. Eigene Darstellung.}
        \end{block}


        
        \begin{block}{8 Farbgestaltung}
            \begin{itemize}
                \item \textbf{Ambiente-Feedback:}
                Farben können als indirekte soziale Signale dienen und den Kontext einer Antwort unterstützen.
                \item \textbf{Kohärenz statt Realismus:}
                Eine konsistente visuelle Umgebung kann Wahrnehmungsinkonsistenzen bei Avataren abmildern.
                \item \textbf{Fokussteuerung:}
                Dezente Farbkontraste helfen, die Aufmerksamkeit auf den Textbereich zu lenken und Ablenkung zu reduzieren.
            \end{itemize}
        \end{block}

        

        \begin{block}{9 Altersabhängige Designpräferenzen}
            Designpräferenzen für Avatare variieren stark je nach Alter und Nutzer-Typ:
            
            \vspace{0.1cm}
            \centering
            
            \includegraphics[width=0.4\linewidth]{Poster/Grafiik2-1.pdf}
            \par\vspace{0.1cm}
            \textit{\footnotesize Abbildung 4: Designpräferenzen. Eigene Darstellung.}
        \end{block}

        \vfill
		
		% 
        \begin{block}{10 Evaluation und Fazit}
            \begin{itemize}
                \item Abhängigkeit von Kontext und Altersgruppe.
                \item Der Uncanny-Valley-Effekt kann durch einen zu hohen Realitätsgrad auftreten
                \item Avatare bieten in textbasierten Chats wenig Mehrwert, sind jedoch bei sprachbasierter Interaktion deutlich sinnvoller.
            \end{itemize}
        \end{block}

        \vfill % Dehnt den Raum
        
        % --- REFERENZEN aus .bib Datei (kompakt) ---
        {\tiny\color{gray}
        \nocite{HerbstRothMerlau2026ChatbotInterfaces}
        \printbibliography[heading=none]
        }


        \vspace{0cm}


    \end{column}
\end{columns}


% Hier ist kein Footer-Code mehr nötig, das erledigt das Template automatisch.

\end{frame}
\end{document}s